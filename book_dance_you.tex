\documentclass[12pt,a4paper]{book}
\usepackage{fontspec}
\usepackage{polyglossia}
\setmainfont{Noto Serif}
\setmainlanguage{russian}
\setotherlanguage{english}

\usepackage{microtype}            % Improves typography (e.g., spacing, kerning)

% Page layout
\usepackage[a4paper,margin=2.5cm]{geometry} % Consistent margins

% Graphics and images
\usepackage{graphicx}             % For including images if needed

% Headers and footers
\usepackage{fancyhdr}
\pagestyle{fancy}
\fancyhf{}                        % Clear default headers/footers
\lhead{\nouppercase{\leftmark}}   % Chapter title in header (no uppercase)
\rhead{\thepage}                  % Page number on right

% Hyperlinks and clickable references
\usepackage[hidelinks]{hyperref}  % Hyperlinks without visible boxes

\setlength{\headheight}{14.5pt} % Fix fancyhdr warning
\pagestyle{fancy}

% Custom environments
\newenvironment{dialogue}{\begin{quote}\itshape\begin{itemize}\item[]}{\end{itemize}\end{quote}}

% Verse spacing adjustment (optional, for better poetry rendering)
\usepackage{setspace}
\AtBeginEnvironment{verse}{\setstretch{1.2}} % Slightly wider line spacing in verse

\newenvironment{innerthought}{\begin{quote}\small\itshape}{\end{quote}}

\usepackage{epigraph}

% Document metadata (placeholders - replace as needed)
\title{Вы танцуете} % Replace with your book title
\author{Vladimir Sumarov} % Replace with your name or pseudonym
\date{\today}

\setcounter{tocdepth}{0}

\begin{document}

\maketitle
\clearpage % Force next page without blank
\begin{center}
    \vspace*{2cm}
    {\Large \textbf{Предупреждение о содержании}} \\
    \vspace{1cm}
\end{center}

\begin{quote}
\textit{«Вы танцуете»} --- это вымышленная история в жанре сюрреалистического триллера, которая может содержать беспокоящие или тревожные элементы, такие как:
\begin{itemize}
    \item Темы абсурда, насилия и психологического напряжения
    \item Ощущения страха, потери контроля и дезориентации
    \item Элементы сюрреализма и метафизического ужаса
    \item Упоминания физического и эмоционального насилия
\end{itemize}

Эти элементы являются частью художественного вымысла и не отражают взглядов автора. \\
\textbf{Читателям рекомендуется проявить осторожность}.
\end{quote}

\clearpage % End on a new page
\tableofcontents
\clearpage

\chapter{Тюрьма с танцами}
\epigraph{«Танец --- это тюрьма, где узник сам себе цепи.»}{Неизвестный автор}

\begin{dialogue}
«Это что, тюрьма с танцами?» \\
«Угу.» \\
«Вечер перестаёт быть томным.»
\end{dialogue}

Сергей окинул взглядом серые стены, освещённые тусклым светом ламп, и услышал, как где-то вдалеке заиграла мелодия --- что-то среднее между вальсом и тюремным шансоном. Его сосед по камере, здоровяк по кличке Борода, лениво кивнул в сторону коридора, откуда доносилась музыка.

\begin{dialogue}
«Новенький, привыкай. Тут каждый вечер так,» --- пробасил Борода, почёсывая щетину. --- «Начальник зоны считает, что танцы --- это реабилитация.»
\end{dialogue}

Сергей хмыкнул. Ещё вчера он сидел в допросной, а сегодня уже оказался в этом странном месте, где вместо звона цепей --- ритм танго. Дверь камеры скрипнула, и в проёме появился охранник с усталым лицом.

\begin{dialogue}
«Выходи, новичок. Твоя очередь на парный танец. Отказываться нельзя, сам понимаешь,» --- сказал он, глядя куда-то мимо Сергея.
\end{dialogue}

Сергей поднялся, бросив взгляд на Бороду, который уже начал напевать что-то про «чёрный ворон». В коридоре его ждала женщина в форме, с суровым взглядом и неожиданно изящной осанкой. Она молча протянула руку.

\begin{dialogue}
«Танго или вальс?» --- спросила она, будто предлагала кофе. \\
«А есть разница?» --- Сергей попытался пошутить, но её взгляд быстро дал понять, что шутки тут не в ходу.
\end{dialogue}

Музыка стала громче, и он, сам того не ожидая, закружился в танце. Стены тюрьмы словно растворились, и на миг показалось, что это не наказание, а какой-то абсурдный спектакль. Но когда мелодия стихла, женщина холодно кивнула и ушла, а Сергей вернулся в камеру, всё ещё слыша ритм в голове.

\begin{dialogue}
«Ну что, реабилитировался?» --- ухмыльнулся Борода, растянувшись на нарах. \\
«Кажется, я теперь ещё больше хочу сбежать,» --- ответил Сергей, глядя в потолок.
\end{dialogue}

За стеной снова заиграла музыка. Вечер обещал быть долгим.

Вечер и правда перестал быть томным ещё до того, как Сергей осознал, куда попал. Он лёг на жёсткую койку, уставившись в трещины на потолке, которые казались ему картой какого-то несуществующего мира. Музыка снаружи не утихала --- теперь это был медленный вальс, пропитанный тоской, словно кто-то играл его на расстроенной гармошке. Борода посапывал рядом, изредка бормоча во сне что-то про «ворон» и «кандалы».

\chapter{Душ как испытание}
\epigraph{«Вода смывает грязь, но не память.»}{Неизвестный автор}

\begin{dialogue}
    «Вечер перестаёт быть томным.»
\end{dialogue}

Сергей ещё посмеивался над этой фразой, прокручивая её в голове, пока шёл по коридору за охранником. Танго с суровой надзирательницей осталось позади, и теперь его вели в душевую --- место, о котором Борода упоминал с какой-то странной ухмылкой. Музыка всё ещё звенела в ушах, смешиваясь с запахом сырости и хлорки. Он вспомнил школьный спортзал: как в седьмом классе его заставляли танцевать на уроках физкультуры, а он, долговязый и неуклюжий, наступал на ноги Ленке из параллельного класса. Она тогда смеялась, а он краснел. Сейчас бы он отдал всё за такую неловкость вместо этого бетонного ада.

Душевая была тесной, с облупившейся плиткой и ржавыми трубами. Пар висел в воздухе, как занавес, скрывая лица. Сергей шагнул внутрь, и дверь за ним с лязгом захлопнулась. Он заметил тени --- три фигуры, вырисовывающиеся сквозь туман. Один из них, невысокий, но жилистый, с татуировкой орла на шее, шагнул вперёд.

\begin{dialogue}
«Эй, русский,» --- голос был хриплым, с акцентом, который Сергей сразу узнал. Кавказский, резкий, как удар ножа. --- «Танцевал там, да? Теперь спляшешь для нас.»
\end{dialogue}

Сергей напрягся. Он не был трусом, но в глазах этих троих читалось что-то звериное. В голове всплыл ещё один флэшбек: школьный двор, где его, пятнадцатилетнего, окружили старшаки с соседнего района.

\begin{innerthought}
«Ты чё, русский, на нашем углу шаришься?» --- вспомнилось ему, и первый удар тогда пришёлся в живот. 
\end{innerthought}
Он отбился, но сейчас бежать было некуда.

\begin{dialogue}
«Чё молчишь?» --- второй, с кривыми зубами и шрамом через бровь, приблизился. --- «У нас тут свой танец. Национальный.»
\end{dialogue}

Сергей попытался отступить, но третий --- громила с руками, как кувалды, --- схватил его за плечо. Влажная плитка под ногами скрипнула, и он почувствовал, как его толкают к стене. Пар обжигал лёгкие, а сердце колотилось, будто снова звучал ритм танго. Он вспомнил надзирательницу --- её холодные пальцы на своей руке, её шаги, чёткие, как метроном.

\begin{innerthought}
«Танго или вальс?» --- эхом отдавалось в голове, пока его прижимали всё сильнее.
\end{innerthought}

Тот, что с орлом, шагнул ближе, дыша ему в лицо смесью табака и кислого пота. Его рука сжала ворот тюремной робы и рванула вниз, ткань затрещала, обнажая кожу. Сергей дёрнулся, но громила ударил его под дых --- коротко, точно, выбивая воздух из лёгких. Он согнулся, хватая ртом пар, а второй, со шрамом, схватил его за волосы и с силой впечатал лицом в кафель. Холод плитки обжёг щеку, кровь из разбитой губы смешалась с водой, стекающей по подбородку. Третий навалился сзади, прижимая его к стене всем весом, пока грубые пальцы впивались в плечи, оставляя синяки.

\begin{dialogue}
«Не дёргайся,» --- прошипел первый, тот, что с орлом. --- «У нас тут свои правила. Ты, славянин, слишком белый для этой зоны.»
\end{dialogue}

Сергей зажмурился, пытаясь уйти в воспоминания. Школьный актовый зал, где он вальсировал с Ленкой под старую пластинку, её тёплая ладонь в его руке, смех одноклассников. Но реальность ворвалась обратно: ещё один удар в рёбра, резкий, ломящий, от которого он чуть не рухнул. Запах пота и крови заполнил ноздри, влажный жар усиливал тошноту. Они не кричали громко --- только короткие выдохи и рычание, как звери над добычей. Его колени подогнулись, он сполз вниз, но их руки держали крепко, не давая упасть полностью. Боль и унижение смешались в одно, пока в ушах звучала та же мелодия --- искажённая, быстрая, как насмешка.

Когда они закончили, его бросили на пол, как тряпку. Вода из душа текла тонкой струйкой, смывая кровь с плитки. Сергей дышал тяжело, глядя в потолок, где ржавчина рисовала странные узоры. В голове крутился вальс --- тот, что он танцевал с Ленкой, и тот, что он только что «сплясал» здесь.

Где-то вдалеке заиграла музыка. Настоящая. Тюремный оркестр снова вступил в свою абсурдную симфонию. Сергей закрыл глаза. Вечер перестал быть томным давно --- теперь он был чёрным, как эта дыра.

\section{Чёрный занавес}

Лёжа на мокрой плитке, Сергей пытался собрать мысли, но они ускользали, как капли воды под пальцами. Боль в теле была тупой, пульсирующей, а в груди росло что-то тяжёлое --- не страх, не стыд, а пустота. Он перевернулся на бок, подтянув колени к груди, и уставился на ржавую трубу в углу. Музыка снаружи звучала приглушённо, словно из другого мира.

\begin{innerthought}
«Это не должно было случиться,» --- подумал он, но мысль оборвалась. Что должно, а что нет, здесь не имело значения.
\end{innerthought}

Шаги охранника прогремели где-то в коридоре, но сюда никто не зашёл. Никто не проверял. Сергей медленно поднялся, опираясь на стену, чувствуя, как дрожат руки. Роба висела лохмотьями, он запахнул её, как мог, и побрёл к выходу. Вечер был чёрным, но ночь обещала быть ещё хуже.

\chapter{Танцующий спаситель}
\epigraph{«Спасение — это шаг в новый ритм, но не всегда в свободу.»}{Неизвестный автор}

Сергей лежал на мокрой плитке душевой, слушая, как вода капает с ржавого крана, а вдалеке тюремный оркестр всё ещё наяривал своё безумное танго. Боль пульсировала в каждом суставе, а в голове крутился вальс из школьных дней --- Ленка, её смех, её тёплая рука. Он почти провалился в забытье, когда услышал шаги. Не тяжёлые, как у охранников, и не угрожающие, как у тех троих. Лёгкие, почти танцующие.

Он приоткрыл глаза. Перед ним стоял человек в длинном плаще, с густой седой бородой и глазами, блестящими, как уличные фонари в тумане. В руках он держал трость с набалдашником в виде звезды.

\begin{dialogue}
«Сергей, сын мой,» --- голос был низким, но мягким, будто из старого радиоспектакля. --- «Я пришёл за тобой.»
\end{dialogue}

Сергей моргнул. Это что, галлюцинация? Он попытался встать, но ноги подкосились. Незнакомец протянул руку, и его пальцы оказались неожиданно сильными.

\begin{dialogue}
«Кто ты?» --- прохрипел Сергей, всё ещё ожидая подвоха. \\
«Зови меня Мастер. Я тот, кто открывает двери,» --- улыбнулся бородач, и в его улыбке было что-то одновременно успокаивающее и жуткое. --- «Пойдём, тюрьма --- не место для танцора.»
\end{dialogue}

Сергей не успел возразить. Мастер щёлкнул пальцами, и душевая растворилась, как дым. В следующую секунду они стояли у чёрного микроавтобуса с тонированными стёклами. Дверь распахнулась, и Сергей, всё ещё в оцепенении, забрался внутрь. Машина рванула с места, а он смотрел на мелькающие за окном огни, думая, что это слишком похоже на сон.

\begin{dialogue}
«Ты волшебник, что ли?» --- спросил он, наконец собравшись с мыслями. \\
\end{dialogue}

Мастер рассмеялся, и его смех эхом отразился от стен салона.

\begin{dialogue}
«Волшебник? О, нет, сын мой. Я лишь проводник. Мы --- Мужской Легион. Братство тех, кто отвергает слабость.»
\end{dialogue}

Сергей нахмурился. В салоне было ещё трое --- крепкие мужчины в чёрных куртках с вышитыми на рукавах эмблемами: кулак, сжимающий молнию. Один из них, с выбритой головой и шрамом на виске, повернулся к нему.

\begin{dialogue}
«Ты прошёл испытание, брат,» --- сказал он, хлопнув Сергея по плечу. --- «Тюрьма закалила тебя. Теперь ты с нами.»
\end{dialogue}

\begin{innerthought}
«Испытание?» --- Сергей почувствовал, как холод пробежал по спине. --- «Вы что, знали про... душ?»
\end{innerthought}

Мастер кивнул, постукивая тростью по полу.

\begin{dialogue}
«Мы наблюдаем. Мы выбираем. Те, кто ломается, остаются там. Ты выстоял.»
\end{dialogue}

В голове Сергея всё смешалось. Танго, душевая, этот странный «спаситель» --- и теперь какие-то сектанты? Он вспомнил школьные уроки литературы: учительница рассказывала про культы, про манипуляции.

\begin{innerthought}
«Они дают тебе надежду, а потом забирают душу,» --- говорила она.
\end{innerthought}

\begin{dialogue}
«Я не хочу в ваш Легион,» --- выдавил он, пытаясь открыть дверь, но та была заперта. \\
\end{dialogue}

Мастер покачал головой, как будто Сергей был капризным ребёнком.

\begin{dialogue}
«Слишком поздно, сын мой. Ты уже танцуешь наш танец.»
\end{dialogue}

Микроавтобус свернул на тёмную дорогу, и из колонок вдруг заиграла музыка --- тот самый вальс, что он танцевал с Ленкой. Но теперь мелодия была искажённой, с металлическим привкусом, будто кто-то добавил в неё барабаны и крики. Мужчины в салоне начали напевать что-то низкими голосами, и Сергей понял, что его «спасение» --- это лишь новый круг ада.

За окном мелькнула вывеска: «Добро пожаловать в Легион». Машина остановилась. Дверь открылась, и его вытолкнули наружу, прямо в толпу таких же «братьев», чьи глаза горели фанатичным огнём. Танцы закончились. Началась новая игра.

\section{Дорога в Легион}

Сергей стоял у микроавтобуса, чувствуя, как холодный ночной воздух обжигает лёгкие. Толпа вокруг него гудела, но он не слышал слов --- только ритм, который всё ещё звучал в голове. Мастер вышел следом, постукивая тростью, и обвёл взглядом своих «братьев».

\begin{innerthought}
«Это не спасение,» --- подумал Сергей, глядя на эмблемы на куртках. --- «Это ловушка. Но как выбраться, если я уже внутри?»
\end{innerthought}

Он оглянулся на микроавтобус, но тот уже растворился в темноте, как мираж. Впереди маячила огромная арка с надписью «Мужской Легион», освещённая тусклыми фонарями. Шаги «братьев» за спиной подталкивали его вперёд, и он пошёл, не зная, куда приведёт этот новый танец.

\chapter{Танцевальный баттл}
\epigraph{«Докажи силу ногами, или она ничего не стоит.»}{Неизвестный автор}

Сергей стоял посреди огромного зала, окружённого десятками мужчин в чёрных куртках с эмблемами кулака и молнии. Потолок был увешан гирляндами, мигающими в ритме какого-то дикого ремикса на вальс, а в центре возвышалась сцена с микрофоном и огромным баннером: «Мужской Легион — сила в движении». Мастер, всё ещё в своём плаще, хлопнул в ладоши, и толпа загудела.

\begin{dialogue}
«Братья!» --- провозгласил он, театрально взмахнув тростью. --- «Сегодня наш новичок, Сергей, докажет свою мужественность... в танцевальном баттле!»
\end{dialogue}

Сергей моргнул. Это что, серьёзно? После всего --- тюрьмы, душевой, этого странного «спасения» --- его заставляют танцевать? Один из «братьев», лысый громила с татуировкой «Сила» на шее, шагнул вперёд и начал нелепо дёргаться под музыку, будто пытался изобразить казачок вперемешку с брейк-дансом. Толпа взревела от восторга.

\begin{dialogue}
«Твоя очередь, новобранец!» --- крикнул Мастер, подмигнув так, что Сергею захотелось его придушить.
\end{dialogue}

Он нехотя вышел на сцену. В голове крутилась мысль: это слишком абсурдно, чтобы быть правдой. Тюремные танцы, сектанты, теперь это --- может, он вообще не в реальном мире? Он вспомнил, как Мастер щёлкнул пальцами в душевой, и всё растворилось.

\begin{innerthought}
«Что, если это не просто фокус, а другая реальность? Может, я сплю, умер или попал в какой-то извращённый эксперимент?»
\end{innerthought}

Музыка заиграла громче --- тот самый вальс с Ленкой, но теперь с басами и синтезатором. Сергей начал двигаться, сначала неуклюже, потом вспоминая школьные уроки. Толпа загудела, кто-то даже крикнул:

\begin{dialogue}
«Давай, брат, жги!»
\end{dialogue}

Он закружился, но в каждом шаге искал подвох. Свет мигнул, и на миг ему показалось, что стены зала дрогнули, как голограмма.

\begin{innerthought}
«Ты видишь?» --- прошептал он сам себе, глядя на Мастера.
\end{innerthought}

Тот стоял с улыбкой, но его глаза были пустыми, как у манекена. После «баттла» (Сергей, к своему удивлению, победил, когда лысый споткнулся о собственные ноги) его отвели в «комнату посвящения». Там стоял стол с пластиковыми стаканчиками и дешёвым пивом, а на стене висел плакат: «Танец — это дисциплина мужчины». Братья начали петь какой-то гимн про «стальные шаги» и «ритм победы», а Сергей всё больше убеждался: это не секта. Это слишком нелепо даже для секты.

Он подошёл к окну, притворяясь, что любуется видом. За стеклом была тьма, но в отражении он заметил, как один из «братьев» на секунду исчез, словно пиксельный сбой. Сердце заколотилось. Он вспомнил фантастические книжки, которые читал в школе: про параллельные миры, про симуляции.

\begin{innerthought}
«Что, если Легион — это не просто сборище чокнутых, а программа, которая тестирует меня? Или хуже — ловушка, где танцы переписывают мой разум?»
\end{innerthought}

\begin{dialogue}
«Мастер,» --- тихо позвал он, обернувшись. --- «Это что, игра?» \\
\end{dialogue}

Мастер замер, а потом рассмеялся, но смех был странным --- слишком ровным, как запись.

\begin{dialogue}
«Игра? О, сын мой, это жизнь! Танцуй или проиграешь!»
\end{dialogue}

Сергей решил проверить теорию. Он схватил трость Мастера, лежавшую на столе, и ударил ею по стене. Раздался треск, и часть стены рассыпалась в искры, обнажая чёрное ничто. Толпа «братьев» замерла, их лица застыли, как у кукол. Мастер шагнул к нему, но его движения были рваными, будто у робота с плохой прошивкой.

\begin{dialogue}
«Ты не должен был этого делать,» --- сказал он, и голос его стал металлическим.
\end{dialogue}

Сергей рванул к разлому в стене, слыша, как за спиной музыка превращается в хаотичный шум. Он прыгнул в темноту, ожидая падения, но вместо этого оказался... в школьном актовом зале. Ленка стояла перед ним, держа руку так, как в том вальсе.

\begin{dialogue}
«Ты вернулся,» --- сказала она, но её голос был эхом. --- «Или ещё нет?» \\
\end{dialogue}

Снаружи послышался топот «братьев». Сергей понял: это не конец. Это только начало следующего танца.

\section{Разлом в реальности}

Сергей замер в школьном актовом зале, глядя на Ленку. Её фигура казалась слишком чёткой на фоне выцветших стен, а эхо её голоса всё ещё висело в воздухе. Топот «братьев» приближался, как барабанная дробь, но он не двинулся с места.

\begin{innerthought}
«Если это игра, то я найду правила,» --- подумал он, сжимая кулаки. --- «А если сон — проснусь.»
\end{innerthought}

Он шагнул к Ленке, но не для танца, а чтобы проверить её. Протянул руку, ожидая, что она растает, как мираж. Но её пальцы были тёплыми, реальными --- или это только казалось? За окнами зала мелькнули тени «братьев», и музыка снова заиграла, слабая, но настойчивая. Сергей понял: следующий шаг будет решающим.

\chapter{Большой Танцевальный Цирк}
\epigraph{«В цирке все танцуют, даже если не хотят.»}{Неизвестный автор}

Сергей прыгнул в разлом в стене, ожидая свободы, но вместо этого пол под ногами мягко прогнулся, как резина, и он рухнул прямо на середину циркового манежа. Прожектора ослепили его, а воздух наполнился запахом попкорна и звериного мускуса. Вокруг раздался рёв толпы, но вместо «братьев» Легиона он увидел клоунов с размалёванными лицами, жонглирующих горящими булавами.

\begin{dialogue}
«Добро пожаловать в Большой Танцевальный Цирк!» --- прогремел голос из динамиков, и Сергей узнал интонации Мастера, только теперь они звучали как у конферансье на стероидах.
\end{dialogue}

Перед ним возникла Ленка --- в ярко-красном трико, с трапецией в руках. Её глаза были пустыми, как у куклы, но она улыбнулась и сказала:

\begin{dialogue}
«Танцуй, Серёжа. Здесь все танцуют.»
\end{dialogue}

Музыка заиграла --- на этот раз что-то среднее между цирковым маршем и школьным вальсом, но с визгом труб и стуком копыт. Сергей попытался бежать, но ноги сами понесли его в центр арены. Он закружился, нелепо размахивая руками, а клоуны начали подбрасывать его в воздух, как акробата. Каждый раз, когда он приземлялся, сцена менялась: вот он на канате, вот в клетке с тиграми, которые рычали в ритм мелодии.

\begin{innerthought}
«Это не реально,» --- пробормотал он, уворачиваясь от горящей булавы. --- «Это сон. Или симуляция. Или чёрт знает что.»
\end{innerthought}

Но прежде чем он успел додумать, свет мигнул, и он оказался в новом месте --- на дискотеке 80-х. Зеркальный шар крутился над головой, а вокруг него отплясывали «братья» Легиона в блестящих костюмах и с начёсами. Мастер, теперь в кожаной куртке и с микрофоном, пел что-то про «ритм мужества» под синтезаторные биты. Ленка стояла у барной стойки, потягивая коктейль с зонтиком, и махнула ему рукой.

\begin{dialogue}
«Давай, Серёжа, зажги!» --- крикнула она, но её голос был искажён, как будто кто-то крутил магнитофон на перемотке.
\end{dialogue}

Сергей попытался остановиться, но ноги снова предали его. Он закружился в нелепом диско, окружённый «братьями», которые выкрикивали:

\begin{dialogue}
«Бери выше! Мужская мощь в каждом па!»
\end{dialogue}

Он заметил, что их движения повторяются, как зацикленная анимация, а лица начинают расплываться, словно маски тают.

\begin{innerthought}
«Это не люди,» --- прошептал он. --- «Это декорации.»
\end{innerthought}

Свет опять мигнул, и вот он уже в балетном зале. «Братья» в пачках и пуантах исполняли пируэты с мрачной сосредоточенностью, а Мастер, теперь в трико и с указкой, командовал:

\begin{dialogue}
«Плие! Гранд жете! Мужественнее, господа!»
\end{dialogue}

Ленка кружилась рядом, но её движения были рваными, как у сломанной балерины из музыкальной шкатулки. Она подошла к нему и прошептала:

\begin{dialogue}
«Ты не выберешься, пока не перестанешь танцевать.» \\
«А как перестать?» --- прохрипел он, пытаясь удержать равновесие на скользком паркете.
\end{dialogue}

Она не ответила --- её фигура рассыпалась в искры, и зал снова сменился. Теперь он стоял на школьной сцене, где впервые танцевал с ней в седьмом классе. Но вместо одноклассников вокруг были «братья», хлопающие в ладоши, а Мастер сидел в первом ряду с блокнотом, ставя оценки.

Сергей упал на колени, задыхаясь. Музыка гремела, перескакивая с вальса на цирковой марш, потом на диско и обратно. Он смотрел на свои руки --- они дрожали, но не от усталости, а от странного ощущения: кожа казалась слишком гладкой, как у манекена.

\begin{dialogue}
«Это не я,» --- сказал он вслух. --- «Это не мой мир.»
\end{dialogue}

Где-то за кулисами послышался топот. «Братья» приближались, их шаги звучали как барабанная дробь. Ленка появилась снова, но теперь её лицо было размытым, как плохая фотография. Она протянула руку и сказала:

\begin{dialogue}
«Танцуй со мной, или они тебя раздавят.»
\end{dialogue}

Сергей сжал кулаки. Он устал от этого бреда. Если это сон, симуляция или ад --- пора заканчивать. Он шагнул к Ленке, но вместо танца схватил её за плечи и крикнул:

\begin{dialogue}
«Кто ты? Что это за место?»
\end{dialogue}

Её глаза мигнули, как экран, и она исчезла. Свет погас. Тишина. А потом --- новый ритм, слабый, но нарастающий. Он понял: это ещё не конец.

\section{Смена декораций}

Сергей стоял в темноте, чувствуя, как новый ритм пульсирует в груди. Свет не возвращался, но он ощущал, что пространство вокруг меняется --- запах попкорна сменился сыростью, шаги «братьев» затихли, уступив место далёкому гулу. Он вытянул руки, пытаясь нащупать стены, но пальцы встретили только пустоту.

\begin{innerthought}
«Если это не конец, то что дальше?» --- подумал он, напрягая слух. --- «Ещё один танец? Или выход?»
\end{innerthought}

Тишина длилась недолго. Ритм вернулся, и с ним --- слабый свет, пробивающийся откуда-то сверху. Сергей поднял голову, но увидел лишь тени, мелькающие в этом свете, как фигуры на экране старого проектора. Он ждал, зная, что следующий шаг снова утянет его в этот безумный хоровод.

\chapter{Хаос против порядка}
\epigraph{«Танец — это порядок, но хаос ломает цепи.»}{Неизвестный автор}

Тишина длилась недолго. Ритм вернулся --- сначала слабый, как стук сердца, потом громче, превращаясь в оглушительный марш. Сергей стоял в темноте, сжимая кулаки, пока свет не вспыхнул снова. Он оказался в огромном ангаре, где «братья» Легиона выстроились в шеренги, маршируя под командой Мастера. Тот стоял на возвышении, размахивая тростью, как дирижёр, а за его спиной висел экран с надписью: «Дисциплина через движение».

Ленка появилась рядом, но теперь она выглядела иначе --- не как кукла из кошмара, а как живая, с усталыми глазами и сжатыми губами. Она схватила Сергея за руку и прошептала:

\begin{dialogue}
«Это не симуляция. Это их мир. Но мы можем его сломать.» \\
«Кто ты вообще?» --- рявкнул он, вырываясь. --- «То ты часть этого бреда, то...» \\
«Я была как ты,» --- перебила она. --- «Они поймали меня раньше. Я сбежала, но не до конца. Помоги мне, и мы оба выберемся.»
\end{dialogue}

Сергей посмотрел на неё, потом на «братьев», которые топали в унисон, их лица --- маски без эмоций. Он вспомнил цирк, дискотеку, балет --- весь этот абсурд. Если это их сила, то и слабость тоже.

\begin{dialogue}
«Что делать?» --- спросил он, чувствуя, как в нём закипает злость. \\
«Танцуй,» --- сказала Ленка. --- «Но не их танец. Наш.»
\end{dialogue}

Мастер заметил их и ткнул тростью в воздух:

\begin{dialogue}
«Новобранец! На сцену! Докажи свою преданность!»
\end{dialogue}

«Братья» загудели, хлопая в ладоши, но Сергей уже знал, что делать. Он шагнул вперёд, но вместо их марша начал двигаться так, как танцевал с Ленкой в школе --- неловко, но искренне. Ленка присоединилась, и их вальс выглядел нелепо на фоне чётких строевых шагов Легиона.

\begin{dialogue}
«Что это за ересь?» --- проревел Мастер, но в его голосе мелькнула растерянность.
\end{dialogue}

Сергей ускорил темп, добавляя хаотичные движения из цирка и диско. «Братья» замерли, их синхронность начала ломаться. Один споткнулся, другой закружился не в ту сторону. Ленка подхватила идею, прыгая и размахивая руками, как в том кошмарном балете.

\begin{dialogue}
«Они не выдержат хаоса!» --- крикнула она. --- «Их система — это порядок!»
\end{dialogue}

Сергей ухмыльнулся. Он вспомнил тюрьму, душ, унижение --- и вложил всю злость в движение. Он прыгнул на одного из «братьев», сбив его с ног, и крикнул:

\begin{dialogue}
«Танцуйте, как хотите, а не как велят!»
\end{dialogue}

Это сработало, как спичка в сухой траве. «Братья» начали дёргаться, кто-то завопил и пустился в дикий пляс, кто-то рухнул, схватившись за голову. Мастер орал, пытаясь восстановить контроль, но его трость треснула пополам, а экран за спиной заискрил и погас.

Ленка схватила Сергея за руку и потащила к стене ангара.

\begin{dialogue}
«Там выход!» --- крикнула она, указывая на трещину, похожую на ту, что он видел раньше.
\end{dialogue}

Они рванули к ней, пока ангар рушился под напором хаотичных танцев «братьев». Кто-то пел шансон, кто-то крутил сальто, а Мастер вопил, падая с возвышения. Сергей и Ленка прыгнули в трещину, и мир вокруг взорвался белым светом.

Когда он открыл глаза, они лежали на траве под открытым небом. Вокруг --- ни ангара, ни Легиона, только тишина и запах дождя. Ленка поднялась, отряхивая грязь с колен.

\begin{dialogue}
«Мы сделали это,» --- сказала она, но её голос дрожал. --- «Или нет?»
\end{dialogue}

Сергей посмотрел на горизонт. Там, вдалеке, мелькнул силуэт Мастера, но он был прозрачным, как мираж. Музыка --- слабый отголосок вальса --- доносилась из ниоткуда. Он сжал руку Ленки.

\begin{dialogue}
«Если это не конец,» --- сказал он, --- «мы станцуем ещё раз. Но уже по-нашему.»
\end{dialogue}

\section{Танец на траве}

Сергей и Ленка лежали на траве, глядя в небо, где облака медленно плыли, как декорации, забытые режиссёром. Тишина была хрупкой, нарушаемой лишь слабым эхом вальса, которое то ли звучало в воздухе, то ли осталось в их головах. Он повернулся к ней, чувствуя тепло её руки в своей.

\begin{innerthought}
«Мы выбрались,» --- подумал он, но уверенности не было. --- «Или просто сменили сцену?»
\end{innerthought}

Силуэт Мастера на горизонте не двигался, но его присутствие ощущалось, как тень, которую нельзя стряхнуть. Сергей поднялся, помогая Ленке встать. Они стояли молча, ожидая, что тишина вот-вот лопнет, и новый ритм снова утянет их в игру.

\chapter{Ловушка сопротивления}
\epigraph{«Каждый бунт — лишь новый шаг в их танце.»}{Неизвестный автор}

Сергей сжал руку Ленки, глядя на горизонт, где маячила прозрачная фигура Мастера. Ветер нёс слабый отголосок вальса, и трава под ногами казалась слишком мягкой, почти нереальной. Он повернулся к Ленке, её лицо было бледным, глаза блестели тревогой.

\begin{dialogue}
«Мы выбрались, да?» --- спросил он, но голос дрожал, выдавая сомнение.
\end{dialogue}

Ленка не ответила сразу. Она посмотрела на Мастера, потом на свои руки, словно проверяя, настоящие ли они.

\begin{dialogue}
«Идём,» --- наконец сказала она, шагая вперёд. --- «Надо проверить.»
\end{dialogue}

Они двинулись к горизонту, где фигура Мастера становилась чётче с каждым шагом. Вальс усиливался, смешиваясь с маршем Легиона, с цирковыми трубами, с синтезаторными битами дискотеки. Сергей чувствовал, как ноги тяжелеют, будто кто-то привязал к ним невидимые гири. Он оглянулся --- трава осталась позади, теперь под ногами был бетон ангара.

\begin{dialogue}
«Нет,» --- прошептал он, останавливаясь. --- «Не опять.»
\end{dialogue}

Ленка обернулась, и её взгляд был полон боли.

\begin{dialogue}
«Я пыталась, Серёжа,» --- сказала она тихо. --- «Я думала, если сломать их порядок, мы вырвемся. Но это ловушка. Каждый бунт — часть их игры.»
\end{dialogue}

Мастер приблизился, его плащ развевался, хотя ветра не было. Он щёлкнул пальцами, и воздух задрожал. Ангар вернулся --- тот самый, с шеренгами «братьев», но теперь их лица были знакомыми. Сергей увидел себя --- десятки Сергеев, марширующих в унисон. Рядом с ними --- копии Ленки, кружащиеся в балетных па с пустыми глазами.

\begin{dialogue}
«Вы не можете уйти,» --- голос Мастера был холодным, как металл. --- «Танец — это вы. Вы — это танец.»
\end{dialogue}

Сергей рванулся к нему, но ноги предали --- он закружился в вальсе, против воли. Ленка схватила его за руку, но её движения зеркалили его, как в кривом зеркале. Он понял: это не просто цикл. Это их суть, их наказание. Тюрьма, душевая, Легион, цирк --- всё было звеньями одной цепи, которую они сами выковали.

\begin{dialogue}
«Почему?» --- прохрипел он, глядя в глаза Мастера.
\end{dialogue}

\begin{dialogue}
«Потому что вы сопротивляетесь,» --- улыбнулся тот. --- «А сопротивление питает нас.»
\end{dialogue}

«Братья» --- их копии --- окружили их, хлопая в ладоши. Музыка стала невыносимой, сжимая голову, как тиски. Сергей упал на колени, Ленка рядом с ним. Их руки сцепились, но пальцы начали таять, растворяясь в ритме.

Последнее, что он увидел, --- её лицо, искажённое криком, который заглушил вальс. Потом темнота. А затем --- новый ритм, слабый, но знакомый. Их шаги эхом отдавались в пустоте. Танец был их цепями, а Мастер --- их тенью. Вальс никогда не кончится.

\section{Эхо в пустоте}

Сергей и Ленка растворились в темноте, но он всё ещё чувствовал её руку --- или её отголосок. Новый ритм пульсировал вокруг, слабый, но неотступный, как дыхание. Он попытался открыть глаза, но видел только черноту, густую, как смола.

\begin{innerthought}
«Это конец?» --- подумал он, но ответа не было. --- «Или просто пауза перед новым кругом?»
\end{innerthought}

Шаги звучали где-то вдалеке --- его или её, он не знал. Музыка, теперь едва слышимая, цеплялась за края сознания, как старая пластинка, которую забыли выключить. Он сжал кулаки, но не почувствовал пальцев. Всё, что осталось, --- ритм, бесконечный и неизбежный.

\chapter{Выход за грань}
\epigraph{«Если мир — это текст, кто пишет конец?»}{Неизвестный автор}

Сергей упал на колени, чувствуя, как музыка сжимает его голову, как Ленка растворяется рядом с ним. «Братья» хлопали, Мастер ухмылялся, и вальс становился вечностью. Но вдруг он замер. Что-то щёлкнуло внутри --- не в ангаре, не в танце, а глубже, в том месте, где мысли перестают быть просто мыслями.

Он поднял голову и посмотрел на Мастера. Тот стоял неподвижно, как статуя, его улыбка застыла. Ленка исчезла, «братья» превратились в серые силуэты, а стены ангара начали дрожать, как плохо нарисованный фон.

\begin{dialogue}
«Это не настоящее,» --- сказал Сергей вслух, и голос его прозвучал громче, чем музыка. --- «Ты не настоящий.»
\end{dialogue}

Мастер моргнул --- впервые за всю историю --- и ответил:

\begin{dialogue}
«А ты кто? Если я не настоящий, то кто ты?»
\end{dialogue}

Сергей встал, игнорируя тяжесть ног. Он шагнул к Мастеру, но тот отступил, и его плащ смялся, как бумага. Ангар задрожал сильнее, и в воздухе повис странный звук --- не мелодия, а что-то похожее на шорох страниц.

\begin{dialogue}
«Ты выдумка,» --- сказал Сергей, глядя прямо в его пустые глаза. --- «Это всё выдумка. Тюрьма, танцы, Ленка, Легион — кто-то это написал.»
\end{dialogue}

Мастер рассмеялся, но смех был рваным, как запись, которую зажевало.

\begin{dialogue}
«И что? Если я выдумка, то ты — часть её. Ты не выйдешь, потому что тебя нет вне этих слов.»
\end{dialogue}

Сергей обернулся. Там, где была Ленка, теперь висела пустота --- белое пятно, как чистый лист. Он шагнул к нему, чувствуя, как пол под ногами растворяется. «Братья» исчезли, Мастер сжался в точку, а музыка превратилась в эхо чьего-то дыхания.

\begin{dialogue}
«Если я персонаж,» --- сказал он, глядя в белизну, --- «то я могу уйти. Ты не решишь за меня.»
\end{dialogue}

Он протянул руку к пустоте и почувствовал, как пальцы касаются чего-то твёрдого --- не стены, а края страницы. Он потянул, и мир треснул, как стекло. Свет ослепил его, но он не закрыл глаза.

\begin{dialogue}
«Эй,» --- крикнул он в никуда. --- «Ты, кто там, по ту сторону! Я больше не танцую.»
\end{dialogue}

Белизна сомкнулась вокруг него, и шорох страниц стих. Его шаги затихли где-то за гранью, оставив лишь тишину и смятую строку, которая никогда не будет допис

\end{document}